\documentclass[12pt]{article}
\usepackage{fullpage}

\usepackage{amsmath}
\usepackage{amssymb}
\usepackage{amsthm}

\bibliographystyle{plain}

\begin{document}
\thispagestyle{empty}

\title{0-1 LCP construction}
\author{Karel Brinda \and Gregory Kucherov \and Kamil Salikhov}
\maketitle

{\bf Goal:} 

for given $k$-mer length construct "zero-one" $LCP$ array ($kLCP$) ($KLCP[i] = 0$ if $LCP[i] < k-1$ and $1$ otherwise) using FM-index, without full SA.

{\bf Algorithm:}

Let's backtrack through all $(k-1)$-mers and try to find them in $FM$-index. If at any step we obtain empty $SA$-interval, we stop this branch of backtracking (first heuristic).
If at any moment we have $SA$-interval with only 1 element - we stop this branch (second heuristics).

If we find any $(k-1)$-mer in $FM$-index, and SA-interval is at least of size $2$ - we put $1$'s in corresponding cells of $kLCP$.

And finally we put $0$'s (this means that $LCP[i]$ is less than $k-1$) for all values in $kLCP$ that were not touched during backtracking.

In general case algorithm works in $O(Nk)$ time, where $N$ is size of $SA$.

{\bf Proof of asymptotics:} 

The whole backtracking can be considered as traversing a tree. We can stop in some node $V$ with corresponding SA-interval $(x, y)$ for two reasons: 
1) $x > y$ ($SA$-interval is empty). In this case the parent of $V$ has non-stopping branch (or stopping by second heuristics), the number of non-stopping branches is up to $N$, so the number of stopping nodes is up to $Nk$ (number of nodes in non-stopping branches) multiplied by $4$ (size of alphabet).
2) $x = y$. For each such branch there is an element in SA, so number of nodes in such branches is up to $Nk$ again. 

So, finally we can traverse up to $4Nk + Nk + N = O(Nk)$ vertices in a tree. 

{\bf Experimental results:}

We tested our construction algorithm on drosophilla genome (about 117 millions bases).

BWT construction time is $155$ seconds for .


\begin{table}[!tb]
\caption{
%Values of $\numstrings(\scheme,4,88M,m)$ 
$0-1$ LCP array construction time for drosophilla genome (117 millions bases) for different values of $k$.
\label{tab:constr_time}}
\centering
\begin{tabular}{| c | c | }
  \hline
  k & time (sec) \\ 
  \hline
  15 & 28 \\
  \hline
  20 & 59 \\
  \hline
  25 & 66 \\
  \hline
  30 & 86 \\
  \hline
\end{tabular}
\end{table}

\end{document}